%%% This is beamerLightboardTemplate.tex (c) 2016 by Tom Roby
%%% Initially created for use with lightboards in Fall 2015. 
%%% Very hacky version; someone who knows what they are doing could likely significantly
%%% improve useability. 
%%% REQUIRES beamerBlockStyle.sty file to fix problem with colored backgrounds in blocks; 
%%%
%%%%%%%%%%%%%%%%%%%%%%%%%%%%%%%%%%%%%%%%%%%%%%%%%%%%%%%%%%%%%%%%%%%%%%%%%%%%%%%%%%%%%%%%%
%				   %%% LICENSE BELOW %%%                                % 
% activityLTapplet.tex is licensed under 						% 
% Creative Commons Attribution-ShareAlike 4.0 International License and also a		% 
% Creative Commons Attribution-NonCommercial-ShareAlike 4.0 International License.	% 
%											% 
% You should have received a copy of a license along with this				% 
% work.  If not, see <http://creativecommons.org/licenses/by-sa/4.0/>			% 
% and also <http://creativecommons.org/licenses/by-nc-sa/4.0/>				% 
%%%%%%%%%%%%%%%%%%%%%%%%%%%%%%%%%%%%%%%%%%%%%%%%%%%%%%%%%%%%%%%%%%%%%%%%%%%%%%%%%%%%%%%%%
%
\documentclass[aspectratio=169,xcolor={table,dvipsnames}]{beamer}

%%% DOCUMENT DATA Fill in your own below
\def\crseNo{MATH 9753}
\def\crseTi{Probabilist Categorification}
\def\crseTiSht{ProbCat}
\def\crseSe{}  %% Leave this off so lectures can be resued.  
\author{Tom Roby}

%%% Number things however you like (or not at all), but best to maintain some flexibility
%%% to add or replace lectures late, and to use with a variety of timelines (MWF vs. TuTh
%%% vs. summer).  Best to tie them to topics or collections of related outcomes, rather than
%%% any particular calendar or even textbook for best reuseability.
\title[VL\#1a: Intro to ProbCat]{Video Lecture \#1a: Introduction to Probabilistic
Categorification (Beamer template for lightboard use)}
\date{}  %% Leave this off so lectures can be resued.  
\setlength{\parskip}{\medskipamount}

%%% DOCUMENT PARAMETERS, e.g., margins, date format note used here
%%% Beamer has its own commands for resetting margins: see that chunk of preamble
% \setlength{\oddsidemargin}{0in}
% \setlength{\evensidemargin}{0in}
% \setlength{\textwidth}{4.5in}
% \setlength{\topmargin}{0in}
% \setlength{\textheight}{9in}

%%% VARIOUS PACKAGES I NEED:

%% Various AMS packages for math
\usepackage{amsfonts}
\usepackage{amsmath}
\usepackage{amssymb}
\usepackage{amsthm}

\usepackage{array}
\usepackage{setspace}


%%% FOR COLOR  BEAMER LOADS ``xcolor'' by DEFAULT, SO OPTIONS NEED TO BE PASSED IN
%%% \documentclass option argument above!!!
%%\usepackage[dvipsnames]{xcolor}
%%% CF: http://cloford.com/resources/colours/500col.htm

\definecolor{myred}{RGB}{220,20,60}  %%% Crimson
\definecolor{mypink}{RGB}{255,62,150} %%% violetred
\definecolor{myblue}{RGB}{100,149,237}   %%% cornflowerblue
\definecolor{mypurple}{RGB}{155,48,255}  %%% purple 1
\definecolor{mygreen}{RGB}{0,201,87}   %%% emeraldgreen
\definecolor{myviolet}{RGB}{218,112,214}  %%% orchid
\definecolor{mygold}{RGB}{255,215,0}  %%% gold
\definecolor{sgibeet}{RGB}{142,56,142}  %%% sgibeet %%% This is the background to frametitles, etc. 
\definecolor{myindig}{RGB}{75,0,130}  %%% plum %%% This is the background to block backgrounds

\newcommand{\cred}[1]{{\color{myred}#1}}
\newcommand{\cpink}[1]{{\color{mypink}#1}}
\newcommand{\cblu}[1]{{\color{myblue}#1}}
\newcommand{\cpurp}[1]{{\color{mypurple}#1}}
\newcommand{\cgrn}[1]{{\color{mygreen}#1}}
\newcommand{\cvio}[1]{{\color{myviolet}#1}}
\newcommand{\cgold}[1]{{\color{mygold}#1}}
\newcommand{\cbeet}[1]{{\color{sgibeet}#1}} %%%Same as for frametitles!
\newcommand{\cindig}[1]{{\color{myindig}#1}} %%%Same as for block backgrounds

%%% END INITIAL COLOR DEFINITIONS, BUT MORE BELOW FOR BEAMER. 

%%% So as to not run out of memory
\usepackage{etex}

%%% For graphics:
% \begin{center}
% \includegraphics[width=4.7in]{PM32BLtoRfinal}
% \end{center}
\usepackage{graphicx}
%% This allows for fancy headers and footers if use command below.  Set
%% to \pagestyle{plain} if don't want.  
% \usepackage{fancyhdr}
% \pagestyle{fancy}
%%% For adding program listings with ``lstlisting'' env.
\usepackage{listings}
%%% To get \dbend sign
\usepackage{manfnt}
\usepackage{svg}
%%% For \newmoon, etc. 
\usepackage{wasysym}
%%% For simple drawing of diagrams, posets: 
\usepackage[all]{xy}
\usepackage{textpos}

%%% Defines various math operators and other notation I use frequently. 
%%% I eliminated most of these, since most people have their own or won't want mine; 
 \def\EE{\mathop{{}\mathbb{E}}}
 \def\PP{\mathop{{}\mathbb{P}}}

%
\def\zz{\mathbb}
\def\bfit#1{{\textit{\textbf{#1}}}}
\def\TE{\exists}
\def\eset{\emptyset}
\def\bull{\noindent $\bullet$\kern 2em}
\def\clickq{\textbf{ClickQuest:  }}
\def\inv{\mathop{\rm inv}}
\def\amaj{\mathop{\rm{amaj}}}
\def\bc#1#2{\left(\kern -2pt{#1\atop #2} \kern -2pt\right)}
\def \bangle{ \atopwithdelims \langle \rangle}
%\def\mchoose{{\left(\atopwithdelims()\right)}}
%%Temporary, until I get a better solution from Ira or Richard
\def\mchoose{\atopwithdelims\langle \rangle}
\def\partitions{\vdash}
\def\bij{\;\longleftrightarrow\;}
\def\onlyif{\quad \Longleftarrow \quad }
%%% Above are TeX-style with \def; Below with LaTeX-style \newcommand
\newcommand{\chk}{{\sc \bf X }}
\newcommand{\divides}{\mid }
\newcommand{\leg}[2]{\left(\frac{#1}{#2}\right) }
\newcommand{\ds}{\displaystyle}
 \def\Bern{\mathop{\rm Bern}}
 \def\Bin{\mathop{\rm Bin}}
 \def\Pois{\mathop{\rm Pois}}
\def\Geom{\mathop{\rm Geom}}
\def\NBin{\mathop{\rm NBin}}
\def\NormDist{\mathop{\mathcal{N}}}
\def\Unif{\mathop{\rm Unif}}
\def\Exp{\mathop{\rm Exp}}
\def\del{\partial}
\newcommand{\RR}{\mathbb R}
%


%%% With amsthm package, creates environments for nicely formatted,
%%% labeled, and numbered propositions, etc. 
\newtheorem{thm}{Theorem}
% \newtheorem{lemma}[thm]{Lemma}
\newtheorem{propo}[thm]{Proposition}
\newtheorem{conj}[thm]{Conjecture}
\newtheorem{cor}[thm]{Corollary}
\newtheorem{eg}[thm]{Example}
\newtheorem{defn}[thm]{Definition}
\newtheorem{rem}[thm]{Remark}
\newtheorem{key}[thm]{Key Idea}
%\newtheorem{fact}[thm]{Fact}
\newtheorem{observ}[thm]{Observation}
\newtheorem{claim}[thm]{Claim}
\newtheorem{open}[thm]{Open Problem}
\newtheorem{prob.}[thm]{Problem}
\newtheorem{quest}[thm]{Question}
\newtheorem{princ}{Principle}
\newtheorem{blank}{}
%%% Various single alphabet characters in crazy math fonts:


%%%%%%%%%%%%%%% BEAMER SPECIFIC SETTINGS %%%%%%%%%%%%%%% 

%%% PICK A BEAMER THEME! 
\usetheme{Frankfurt}

\usefonttheme[onlylarge]{structurebold}
\setbeamerfont*{frametitle}{size=\normalsize,series=\bfseries}
\setbeamertemplate{navigation symbols}{}
%%Get's rid of black bar across the top for sections\dots 
\setbeamertemplate{headline}{}
%\setbeamertemplate{section in head/foot shaded}[default][60]
%\setbeamertemplate{subsection in head/foot shaded}[default][60]
%% Changed first opaqueness value to <0> from <1>
\beamersetuncovermixins{\opaqueness<0>{25}}{\opaqueness<2->{15}}

%%% RECOMMENDED BY STACK EXCHANGE http://tex.stackexchange.com/questions/26476/add-footer-text-to-all-slides-in-beamer
\setbeamertemplate{footline}[text line]{%
  \parbox{0.4\linewidth}{
    \vspace*{-8pt}\color{\footerColor}\crseNo~(\crseTiSht) %, ~\crseSe 
  }\
  \parbox{0.45\linewidth}{
    \vspace*{-8pt} \color{\footerColor}\insertshorttitle~(\insertshortauthor) 
  }
  \hfill%
  \parbox{0.15\linewidth}{
    \vspace*{-8pt}\raggedleft {\color{\footerColor}\insertframenumber{}}
    {\color{\footerColor}/ \inserttotalframenumber\hspace*{1ex} }
%%\insertpagenumber
%    \vspace*{-8pt}\raggedleft\insertpagenumber
  }
}


\setbeamercolor{background canvas}{bg=black}
\def\footerColor{mypink}
\setbeamercolor*{frametitle}{fg=white,bg=sgibeet}
\setbeamercolor*{title}{fg=white,bg=sgibeet}
\setbeamercolor{normal text}{fg=white,bg=black!90}
\setbeamercolor*{block title}{fg=white,bg=sgibeet}
\setbeamercolor*{item}{fg=sgibeet, bg=white}
\setbeamercolor*{block body}{fg=white,bg=black}


%%% TR (21 Oct. 2015): my attempt to get the frametitle not across the entire width of the page, but still
%%% starting right at the left end of the page; Note the need to pull the colorbox back by
%%% 10pt, then put the text back to 10pt away from the margin.  Also, needed to add 15pt to
%%% \textwidth, which is most easily done with the \dimexpr command (doesn't work otherwise\dots )
\setbeamertemplate{frametitle}{%
\hskip -10pt \begin{beamercolorbox}[wd=\dimexpr\textwidth+15pt\relax, ht=0.5cm, dp=0.2cm]{frametitle}
\hskip 10pt
\usebeamerfont{frametitle}\insertframetitle
\end{beamercolorbox}
}


%%% GOLD for math mode (hence the $$);
\setbeamercolor{math text}{fg=mygold}
\setbeamercolor{math text displayed}{fg=mygold}

%%% This Roby-defined package replaces the contrastive background that beamer uses to ``frame'' proclaimations
%%% (Theorems, Conjectures, etc.) with a thin boarder.  The former is opaque to the
%%% lightboard technology, making it impossible to point at anything inside such a
%%% ``frame''.  But eliminating the frame entirely, made it impossible to see where the
%%% proclamation ended.  
%%% This package created by TR, Sept. 2015
\usepackage{beamerBlockFramed}
\setbeamertemplate{blocks}[framed]
%%% USER SETABLE PARAMETERS for block frames; 
%% This is the width of the border around blocks;
\def\blockFrameBdrWidth{1.4pt} 	   	  %%% Default= 1.4pt
%% This is the color of the border framing a block;
\def\blockFrameBdrColor{sgibeet}         %%% Default=sgibeet, same as frametitle bg


%%% RESETTING MARGINS TO LEAVE SPACE FOR ME TO WRITE AND BE SEEN
\setbeamersize{text margin left=10pt}
\setbeamersize{text margin right=2.3in}


%%% FOR PRINTING HANDOUTS TO BRING TO CLASS: 
%%% UNCOMMENT region below to get black on white with good colors
%%% ALSO, add ``handout'' to the list of optional arguments in the
%%% \documentclass[handout,etc.]{beamer} AT TOP
%%% I suppose the right way to do this is with some LaTeX declarations, but I just use emacs
%%% to comment or uncomment regions with a couple of keystrokes. 

% \setbeamercolor{background canvas}{bg=white}
% \setbeamercolor*{frametitle}{fg=black,bg=Cerulean}
% \setbeamercolor*{title}{fg=black,bg=Cerulean}
% \setbeamercolor{normal text}{fg=black,bg=white}
% \setbeamercolor*{block title}{fg=black,bg=Cerulean}
% \setbeamercolor*{item}{fg=BrickRed, bg=white}
% \setbeamercolor*{block body}{fg=black,bg=white}
% \setbeamercolor*{math text}{fg=OliveGreen}
% \setbeamercolor*{math text displayed}{fg=OliveGreen}
% \def\footerColor{Bittersweet}

% \def\cred#1{{\color{red}#1}}
% \def\cpink#1{{\color{BrickRed}#1}}
% \def\cpurp#1{{\color{mypurple}#1}}
% \def\cgrn#1{{\color{OliveGreen}#1}}
% \def\cgrn#1{{\leavevmode\color{OliveGreen}#1}}
% \def\cvio#1{{\color{violet}#1}}
% \def\cgold#1{{\color{orange}#1}}
% \def\cbeet#1{{\color{sgibeet}#1}} %%%Same as for frametitles!
% \def\cindig#1{{\color{indig}#1}} %%%Same as for block backgrounds

%%% END REGION TO UNCOMMENT FOR PRINTING


\begin{document}


%%% HERE'S A BLANK FRAME

%%%%%%%%%%% FRAME BREAK %%%%%%%%%%%%
\begin{frame}

\titlepage
\end{frame}


%%% ^^^^^^^^^^^^^^^^^^^^^^^^^^^^ CURRENT POINT ^^^^^^^^^^^^^^^^^^^^^^^^^^^^ 

%%%%%%%%%%% FRAME BREAK %%%%%%%%%%%%
\begin{frame}
 \section{}
 \frametitle{Suggestions for Use} 

This is a beamer template suitable for lightboards as described at the incredibly useful
website \url{http://lightboard.info/}.  Go there for all the basics: how to make one,
technological tips, and best practices.  




\end{frame}



%%%%%%%%%%% FRAME BREAK %%%%%%%%%%%%
\begin{frame}
 \section{}
 \frametitle{Color Testing!} 

Which of the following colors to you particularly like or find particularly hard to read?
You can change any of these in the preamble.  

\cred{This is red text!}

\cpink{This is pink text!}

\cgold{This is gold text!}

\cgrn{This is green text!}

\cblu{This is blue text!}

\cindig{This is indigo text (more for background use).}

\cvio{This is violet text!}

\cpurp{This is purple text!}

\end{frame}



%%%%%%%%%%% FRAME BREAK %%%%%%%%%%%%

\begin{frame}
%  \section{}
\frametitle{Outline} 

\cred{\bfit{This is really meant to be a beamer template for lightboard videos.  I'm just
leaving in some sample math from earlier lightboard lectures I made on probability theory.}}

\begin{itemize}
\cblu{

\item Definition of \cred{\emph{moment generating
function}}; 

\pause

\item The MGF determines a RV uniquely; 

\pause

\item Examples for standard distributions; 

\pause

\item Applications; 
}
\end{itemize}


%\vspace{3in}

\end{frame}

%%% ^^^^^^^^^^^^^^^^^^^^^^^^^^^^ CURRENT POINT ^^^^^^^^^^^^^^^^^^^^^^^^^^^^ 
%%%%%%%%%%% FRAME BREAK %%%%%%%%%%%%
\begin{frame}
%  \section{}
  \frametitle{Moment Generating Functions} 

\begin{defn}
The \cred{moment generating function} $m_{X}(t)$ of a RV $X$ is 
\centerline{$ \ds m_{X}(t) = \EE e^{tX}$ \cpink{(whenever defined!)} $ \ds  = \sum_{k=0}^{\infty}\EE [X^{k}]\frac{t^{k}}{k!}$}
\end{defn}
\cblu{You can think of this as the Laplace transform of $X$ (FTWK); or as the exponential
generating function for \cred{\emph{all}} the moments of $X$. 
\pause

\textbf{Discrete:} $=\sum e^{tx}p(x)$; \textbf{cont:} $\int e^{tx}f(x)\,dx$. 
}
\pause

(More advanced: \emph{characteristic function} $\phi_{X}(t) = \EE [e^{itX}]$ always exists
(converges). Analogue of Fourier transform.)
\pause
\begin{propo}
If $m_{X}(t) = m_{Y}(t) \forall\,t\in (a,b)\subseteq \RR $, then $X=Y$ (as RV).   
\end{propo}
So the MGF basically determines the RV.  

\vskip 5in
\end{frame}



%%%%%%%%%%% FRAME BREAK %%%%%%%%%%%%
\begin{frame}
  \section{}
  \frametitle{Some Examples} 
\cblu{Here are MGF for some of our favorite RVs: 
\pause

$\cgrn{X\sim \Bern (p)}$: $m_{X}(t) = e^{0\cdot t}\cdot (1-p) + e^{1\cdot t}\cdot p$
$ = (1-p) + pe^{t} = 1 + pt + p\frac{t^{2}}{2!}+ p\frac{t^{3}}{3!}+\dots $.

\pause

$\cgrn{X\sim \Pois (p)}$: $\ds m_{X}(t) = \sum_{k=0}^{\infty} e^{tk}\left(e^{-\lambda}\frac{\lambda^{k}}{k!}\right)$
$\ds = e^{-\lambda}\sum_{k=0}^{\infty} \frac{(\lambda e^{t})^{k}}{k!} =
e^{-\lambda}e^{\lambda e^{t}} = e^{\lambda (e^{t}-1)}$. 
\pause

$\cgrn{X\sim \Exp (\lambda )}$: $\ds m_{X}(t) = \int_{0}^{\infty} e^{tx}\lambda
e^{-\lambda x}\,dx$
$\ds  = \lambda \int_{0}^{\infty} e^{-(\lambda - t)x}\,dx = \frac{\lambda}{\lambda -t}$. 

\pause

\cpink{Reality Check: Do $m'(0)$ and $m''(0)$ give correct results?}
}
\pause


\vskip 5in
\end{frame}


%%%%%%%%%%% FRAME BREAK %%%%%%%%%%%%
\begin{frame}
  \section{}
  \frametitle{Sums of independent RVs and MGFs} 
\cblu{MGFs behave nicely for sums of RVs: 
\begin{propo}
If $X$ and $Y$ are \emph{independent} RVs, then $\ds m_{X+y}(t) = m_{X}(t)m_{Y}(t)$. 
\end{propo}
\pause

\cgrn{If $X_{i}\sim \Bern (p)$ are independent}: Then $ \ds Y\sim \Bin(n,p) = \sum_{i=1}^{n} X_{i}$; hence,

$\ds m_{Y}(t) = \prod_{i=1}^{n}m_{X_{i}}(t) = \left[ (1-p) + pe^{t} \right]^{n}$.

\pause

\cpink{Reality Check: Do $m'(0)$ and $m''(0)$ give correct results?}
}
\pause


\vskip 5in
\end{frame}

%%%%%%%%%%% FRAME BREAK %%%%%%%%%%%%
\begin{frame}
  \section{}
  \frametitle{Applications} 
\cblu{Independent Poissons: 

If $X\sim \Pois(\lambda )$ and $Y\sim \Pois (\mu )$ are independent, then 
\[
m_{X+Y}(t) = m_{X}(t)m_{Y}(t) = e^{\lambda(e^{t}-1)}e^{\mu (e^{t}-1)} = e^{(\lambda +\mu)
(e^{t}-1)}. 
\]
\pause
Hence, adding independent Poissons gives $\Pois (\lambda +\mu)$.  
\pause

Similar calculations show that for $Z\sim \NormDist (0,1)$

$\ds m_{Z}(t) = e^{t^{2}/2} \implies $\\ 
$\ds m_{\mu +\sigma^{2}Z} = \EE e^{t\mu}e^{t\sigma Z} = e^{t\mu +t^{2}\sigma^{2}/2}$. 

Using this, it's one line to see that if $X\sim \NormDist (a,b^{2})$ and $Y\sim \NormDist
(c,d^{2})$ are independent, then $X+Y\sim \NormDist (a+c,b^{2}+d^{2})$.  
}
\vskip 5in
\end{frame}

%%%%%%%%%%%%%%%% END DOCUMENT %%%%%%%%%%%%%%%%%%%%%%%%%%
\end{document}


%%How to change the fontsize locally in beamer
%\fontsize{10}{11}\selectfont

